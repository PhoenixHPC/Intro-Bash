\setModuleTitle{More Tips And Tricks}
\setModuleAuthors{%
  Stephen Bent, Robinson Research Institute, University of Adelaide\\
  Steve Pederson, Bioinformatics Hub, University of Adelaide \mailto{stephen.pederson@adelaide.edu.au}\\
}
\setModuleContributions{%
  Dan Kortschak, Adelson Research Group, University of Adelaide \mailto{dan.kortschak@adelaide.edu.au} \\
}

%----------------------------------------------------------------------------------------
% MODULE TITLE PAGE
%----------------------------------------------------------------------------------------
% BEGIN: Module Title Page
%  * The chapter page will always appear on odd numbered page
\chapter{\moduleTitle}
\newpage

\section{Text In the Terminal}
We can display a line of text in \texttt{stdout} by using the command \texttt{echo}.
The most simple function that people learn to write in most languages is called \texttt{Hello World} and we'll do the same thing today.
\begin{steps}
\begin{lstlisting}
echo 'Hello World'
\end{lstlisting}
\end{steps}
That's pretty amazing isn't it \& you can make the terminal window say anything you want without meaning it.\\
\begin{lstlisting}
echo 'This computer will self destruct in 10 seconds!'
\end{lstlisting}

\begin{information}
There are a few subtleties about text which are worth noting.
Inspect the \texttt{man echo} page \& note the effects of the \texttt{-e} option.
This allows you to specify tabs, new lines \& other special characters by using the backslash to signify these characters.
This is an important concept \& the use of a backslash to ``escape'' normal meaning of a character is very common.
In the following, we are using the backslash to escape the normal meanings of the \texttt{t} and \texttt{n} characters, and they can take on their ``special" meaning, such as a \texttt{tab} delimiter, or a newline.
Try the following three commands \& see what effects these special characters have.
\end{information}
\begin{steps}
\begin{lstlisting}
echo 'Hello\tWorld'
echo -e 'Hello\tWorld'
echo -e 'Hello\nWorld'
\end{lstlisting}
\end{steps}

\section{Redirection Using The Pipe Symbol}
\begin{information}
A very common process in the command line is to take the output of one process and send it to another.
For example, we might want to cut a column from a csv file with categorical information, and then send that field to have the different categories counted by another tool.
This is where the concept of \texttt{stdout} becomes more important. \\
\end{information}

\begin{steps}
By default, most command-line tools print their output to \texttt{stdout} but instead, we can send this to another tool using the pipe (\texttt{|}) symbol.
This is exactly like putting a pipe on the output of one process, and routing it to the \textit{standard input} or \texttt{stdin} of another.
We can use this to build up a series or chain of processes in a single line.
A slightly ridiculous example might be to print lines 96 to 100 of the file \texttt{words} by combining \texttt{head} and \texttt{tail}
\begin{lstlisting}
head -n100 ~/firstname/words | tail -n5
\end{lstlisting}
\end{steps}

\begin{information}
\textbf{Note that we didn't give a file to \texttt{tail} to work with in the above command.}
It took the \texttt{stdout} from the command \texttt{head} as it's input via \texttt{stdin}.
This is how the pipe will work on virtually all occasions.
\end{information}

\begin{steps}
As another simple example, we could take some long output from the \texttt{ls} command \& send it to the pager \texttt{less}.
\begin{lstlisting}
ls -lh /usr/bin | less
\end{lstlisting}
Page through the output until you get bored, then hit \texttt{q} to quit.
\end{steps}

\section{Sending Output To A File}
\begin{information}
So far, the only output we have seen has been in the terminal, i.e  \texttt{stdout}.
Similar to the pipe command, we can redirect the output of a command \textbf{to a file} instead of \texttt{stdout}, and we do this using the greater than symbol (\textgreater), which we can almost envisage as an arrow.\\

\end{information}

\begin{steps}
To see this in action, we'll collect all the words matching our key pattern, \& write these to a file.
\end{steps}
\begin{lstlisting}
cd ~/firstname
egrep 'ample$' words > ampleEndings.txt
less ampleEndings.txt
\end{lstlisting}

\begin{information}
This is similar to the $>>$ trick we used earlier, but using only a single $>$ symbol will directly overwrite any existing file with the new information.
\textbf{When using the command line, there is no }\texttt{Undo} \textbf{command.}
Once we do something, it cannot be undone!
\end{information}

Once you've had a quick look at the file, exit the less pager and delete the file using the \texttt{rm} command.
\begin{lstlisting}
rm ampleEndings.txt
\end{lstlisting}

\section{Downloading Data}

\begin{steps}
The terminal makes obtaining data from any publicly available sources very easy using the command \texttt{wget}, once you know the location of a file.
Let's find a genome to download by opening Firefox \& heading to \url{http://www.ncbi.nlm.nih.gov/genome/1099}.
This is the page for everyone's favourite probiotic \texttt{Lactobacillus\_acidophilus}.
We'll download the \textit{General Feature Format} file with the suffix \texttt{.gff} file which contains information about any identified genomic features, and the link to this is in the box at the top of the page.
Right-click on the link to the GFF \& select \textit{Copy Link Location}.
Now head back to the terminal , and if you're not already in your personal directory, i.e. \texttt{\~{}/firstname} then change into this directory and type the command \texttt{wget}.
Once you've typed this command, right-click and paste the contents of the clipboard to give the following (apologies for the small font):
\end{steps}

\begin{lstlisting}
wget <<paste address here>>
\end{lstlisting}

This will download the selected file into your current working directory.
Don't forget to have a look at the \texttt{man} page to make sure you understand how the command \texttt{wget} works.

\begin{information}
The file we have downloaded using \texttt{wget} is in simple text format, but it has been compressed for easier transfer.
The suffix \texttt{.gff} really just denoted a specific structure within the text file so that we know where to look within the file for certain information.
Most software that uses a file in this format will manually check for the correct structure first.
However, when working in the bioinformatics field,  you will often see or require compressed files .
This is a common enough concept for even Windows and MAC users, but a few specific compression types are commonly used in bioinformatics.
The main differences are in the compression algorithms used, and that is far beyond the scope of what most of us need to worry about.
\end{information}


\section{Working With Compressed Files}


The most common types of compression you will see are:\\
\begin{center}
	\begin{tabular}{ p{2cm}  p{3cm}  p{3cm}   p{4cm}}
		\toprule
		\textbf{Suffix} & \textbf{Compress} & \textbf{Extract} & \textbf{Useful Arguments} \\
		\midrule
		\texttt{.zip} & \texttt{zip} & \texttt{unzip} & \texttt{-d, -c, -f} \\
		\midrule
		\texttt{.gz} & \texttt{gzip} & \texttt{gunzip} & \texttt{-d, -c, -f} \\
								 &								& \texttt{zcat} & \\
		\midrule
		\texttt{.tar.gz} & \texttt{tar} & \texttt{tar} & \texttt{-x, -v, -f, -z} \\
		\midrule
		\texttt{.bz2} & \texttt{bzip2} & \texttt{bunzip2} & \\
		\bottomrule
	\end{tabular}
\end{center}

Often, files you download this way will be compressed (tar: tape archive) and archived (zipped). 
If you see file name suffixes like .tar, .zip, .gz, and/or .bz2, among others, that is what these are.  
To explore what these command-line options do, please check the \texttt{man} pages. \\

\begin{steps}
Just to demonstrate how we do this on the command line, we could compress one of our fastq files
\begin{lstlisting}
cd ~/firstname
gzip words
ls
\end{lstlisting}
\end{steps}

\begin{questions}
How would we now decompress (or extract) this file?\\
\begin{answer}
\texttt{gunzip words.gz}\\
\end{answer}

How could we have kept our compressed file, along with the decompressed output?\\
\begin{answer}
\texttt{gunzip -k words.gz}\\
\end{answer}

Why would we use \texttt{zcat} instead of \texttt{gunzip -c}?\\
\begin{answer}
It's less typing, and it's easier to read back
\end{answer}
\end{questions}


\begin{steps}
The \texttt{GFF} file has been downloaded as a compressed file using the \texttt{gzip} program so the first thing we need to do is uncompress it.
\begin{lstlisting}
gunzip GCF_000011985.1_ASM1198v1_genomic.gff.gz
\end{lstlisting}
Don't forget to use tab auto-complete!
\end{steps}


\begin{steps}
The first 7 lines of this file is what we refer to as a \textit{header}, which contains important information about how the file was generated in a standardised format.
Many file formats have these structures at the beginning but for our purposes today we don't need to use any of this information so we can move on.
Have a look at the beginning of the file to see what it looks like.
\end{steps}
\begin{lstlisting}
head GCF_000011985.1_ASM1198v1_genomic.gff
\end{lstlisting}

This file begins with a few lines of header information which begin with one or two hash symbols.
These lines contain helpful information about how the file was generated.
The remainder of the file contains information about the genomic features themselves, in tab-separated format.
Each line represents a genomic feature, and the tab-separated values correspond to columns in a spreadsheet.
The first feature is annotated as a \textit{region} in the third field, whilst the second feature is annotated as a \textit{gene}.
We'll come back to this file in the next section.

