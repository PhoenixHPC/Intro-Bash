\setModuleTitle{Writing Scripts}
\setModuleAuthors{%
  Stephen Bent, Robinson Research Institute, University of Adelaide\\
  Steve Pederson, Bioinformatics Hub, University of Adelaide \mailto{stephen.pederson@adelaide.edu.au}\\
}
\setModuleContributions{%
  Dan Kortschak, Adelson Research Group, University of Adelaide \mailto{dan.kortschak@adelaide.edu.au} \\
  Jimmy Breen, Robinson Research Institute \& Bioinformatics Hub, University of Adelaide \mailto{jimmy.breen@adelaide.edu.au}\\
}

%----------------------------------------------------------------------------------------
% MODULE TITLE PAGE
%----------------------------------------------------------------------------------------
% BEGIN: Module Title Page
%  * The chapter page will always appear on odd numbered page
\chapter{\moduleTitle}
\newpage

Sometimes we need to perform repetitive tasks on multiple files, or need to perform complex series of tasks and writing the set of instructions as a script is a very powerful way of performing these tasks.
They are also an excellent way of ensuring the commands you have used in your research are retained for future reference.
Keeping copies of all electronic processes to ensure reproducibility is a very important component of any research. 
Writing scripts requires an understanding of several key concepts which form the foundation of much computer programming, so let's walk our way through a few of them. \\


\section{Shell Scripts}
Now that we've been through just some of the concepts \& tools we can use when writing scripts, it's time to tackle one of our own where we can bring it all together.

\begin{information}
Every bash shell script begins with what is known as a \textit{shebang}, which we would commonly recognise as a hash sign followed by an exclamation mark, i.e \texttt{\#!}.
This is immediately followed by \texttt{/bin/bash}, which tells the interpreter to run the command \texttt{bash } in the directory \texttt{/bin}.
This opening sequence is vital \& tells the computer how to respond to all of the following commands.
As a string this looks like:\\

\texttt{\#!/bin/bash}\\
\end{information}

\begin{note}
The hash symbol generally functions as a comment character in scripts.
Sometimes we can include lines in a script to remind ourselves what we're trying to do, and we can preface these with the hash to ensure the interpreter doesn't try to run them.
It's presence as a comment here, followed by the exclamation mark, is specifically looked for by the interpreter but beyond this specific occurrence, comment lines are generally ignored by scripts \& programs.
\end{note}

\subsection{Some Example Scripts}
Let's now look at some simple scripts.
These are really just examples of some useful things you can do \& may not really be the best scripts from a technical perspective.
Hopefully they give you some pointers so you can get going


\subsubsection*{A Simple Example to Start}
\begin{warning}
\textbf{Don't try to enter these commands directly in the terminal!!!}
They are designed to be placed in a script which we will do after we've inspected the contents of the script.
First,  just have a look through the script \& make sure you understand what the script is doing.

Also remember that any long lines of code are automatically broken into new lines on your page by the `\textbackslash ' character.
We don't to enter this character when we create our script.
Note that the line numbers on the left of the code don't change when this happens, e.g. line 9.
\end{warning}

\begin{lstlisting}[style=command_syntax]
#!/bin/bash

# First we'll declare some variables with some text strings
ME='Put your name here'
MESSAGE='This is your first script'

# Now well place these variables into a command to get some output
echo -e "Hello ${ME}\n${MESSAGE}\nWell Done!"
\end{lstlisting}

\begin{information}
Firstly, you may notice some lines that begin with the \# character.
These are \textit{comments} which have no impact on the execution of the script, but are written so you can understand what you were thinking when you wrote it.
If you look at your code 6 months from now, there is a very strong chance that you won't recall exactly what you were thinking, so these comments can be a good place just to explain something to the future version of yourself.
There is a school of thought which says that you write code primarily for humans to read, not for the computer to understand.
\end{information}

\begin{questions}
In the above script, there are two variables. 
Although we have initially set them to be one value, they are still variables.
What are their names? 
\begin{answer}
ME \& MESSAGE
\end{answer}
\end{questions}

\begin{steps}
Using the text editor \textit{gedit}, enter the above code into a file setting your actual name as the ME variable,  and save it as \texttt{wellDone.sh} in your \texttt{\~{}/firstname} folder.\\
\end{steps}

\begin{information}
You'll notice that once you've done this, the text will become colour coded based on whether the text is a comment, a variable or a shell command.
Scripts are usually saved with the \texttt{.sh} suffix and this colour coding can be very helpful with writing scripts.\\

Another coding style which can be helpful is the enclosing of each variable name in curly braces every time the value is called.
Whilst not being strictly required, this can make it easy for you to follow in the future when you're looking back.
Variables have also been names using strictly upper-case letters.
This is another optional coding style, but can also make things clear for you as you look back through your work.
Most command line tools use strictly lower-case names, so this is another reason the upper-case variable names can be helpful.
\end{information}

\begin{steps}
Unfortunately, this script cannot be executed yet but we can easily enable execution of the code inside the script.
If you recall the flags from earlier which denoted the read/write/execute permissions of a file, all we need to do is set the execute permission for this file.
First we'll look at the files in the folder using \texttt{ls -l} and note these triplets should be \texttt{rw-} for the user \& the group you belong to.
To make this script executable, enter the following in your terminal.
\begin{lstlisting}
cd ~/firstname
chmod +x wellDone.sh
ls -l
\end{lstlisting}
\end{steps}

\begin{steps}
Notice that the third flag in the triplet has now become an \texttt{x}.
This indicates that we can now execute the file in the terminal.
As a security measure, Linux doesn't allow you to execute a script from within the same directory so to execute it enter the following:
\begin{lstlisting}
./wellDone.sh
\end{lstlisting}
\end{steps}

\subsubsection{Making a Small Change}

\begin{steps}
Now let's change the variable \texttt{ME} in the script to read as
\begin{lstlisting}
ME=$1
\end{lstlisting}
and save this as \texttt{wellDone2.sh}.
You'll now need to set the execute permission again.
\begin{lstlisting}
chmod +x wellDone2.sh
\end{lstlisting}
\end{steps}

\begin{information}
This time we have set the script to \textit{receive input from stdin} (i.e. the terminal), and we will need to supply a value, which will then be placed in the variable \texttt{ME}.
Choose whichever random name you want and enter the following
\begin{lstlisting}
./wellDone2.sh Boris
\end{lstlisting}
\end{information}

\begin{advanced}
As you can imagine, this style of scripting can be useful for iterating over multiple objects.
A trivial example, which builds on a now familiar concept would be to try the following.
\begin{lstlisting}
for n in Boris Fred; do (./wellDone2.sh $n); done
\end{lstlisting}
\end{advanced}

\subsubsection*{A Script That Does Something Relevant}
\begin{lstlisting}[style=command_syntax]
#!/bin/bash
FILE=$1

# Find all of the possible read lengths in the supplied file
LENGTHS=$(sed -n '2~4p' ${FILE} | awk '{print length($1)}' | sort | uniq)

# Print some column headings
echo -e "Read length\tTotal Number"

# Now for each possible length, count how many reads there are of that length
for L in ${LENGTHS}
do
	COUNT=$(sed -n '2~4p' ${FILE} | awk '{print length($1)}' | grep -c ${L})
	echo -e "${L}\t${COUNT}" 
done
\end{lstlisting}

\begin{steps}
Now use the text editor \texttt{gedit} to write this script \& save it as \texttt{findLengths.sh} in your home folder.
\end{steps}

\begin{note}
Note that after the shebang, there was a variable \texttt{FILE} which took the value \texttt{\$1}.
This means that after we call the script, we need to specify a filename to assign to the variable \texttt{FILE}.
For example, we could run the script as: \\
\texttt{./findLengths.sh \~{}/Documents/trainingData/seqData.fastq}.\\
This would execute the remainder of the script on that file. \\
\end{note}

\begin{questions}
Where will the script send the output to?
\begin{answer}
stdout.
\end{answer}

How would we write the output to a file?
\begin{answer}
Use the $>$ symbol to assign it to a filename
\end{answer}
\end{questions}

\begin{steps}
Now we can make the script executable \& run it on the file \texttt{seqData.fastq}.
\begin{lstlisting}
chmod +x findLengths.sh
./findLengths.sh ~/Documents/trainingData/seqData.fastq > lengths.txt
\end{lstlisting}
\end{steps}

\begin{questions}
Inspect the file using \texttt{gedit}, or the \texttt{less} pager.
Did it look like you expected? 
Is this a tab-separated file, or a comma or space separated file?
\end{questions}

\begin{bonus}
Try running it on the \texttt{pair1.fq} \& \texttt{pair2.fq} files.
\begin{questions}
Is that what you expected?
\begin{answer}
All the reads should have been the same length...
\end{answer}
\end{questions}
\end{bonus}

\clearpage
\subsection*{A more complicated script}

Here's a more complicated script with some more formal procedures.
This is a script which will extract only the CDS features from the .gff file we have been working with, and export them to a separate file.
Look through each line carefully \& make write down your understanding of what each line is asking the program to do.

\begin{lstlisting}[style=command_syntax]
#!/bin/bash

# Declare some helpful variables
FILEDIR=~/nectar-workshop-template/examples/intro_ubuntu_2016/files
FILENAME=NC_015214.gff
OUTFILE=NC_015214_CDS.txt

# Make sure the directory exists
if [ -d ${FILEDIR} ];
  then
    echo Changing to ${FILEDIR}
    cd ${FILEDIR}
  else
    echo Cannot find directory ${FILEDIR}
    exit 1
fi

# If the file exists, extract the important CDS data
if [ -a ${FILENAME} ];
  then
    echo Extracting CDS data from ${FILEDIR}/${FILENAME}
    echo "SeqID Source Start Stop Strand Tags" > ${OUTFILE}
    awk '{if (($3=="CDS")) print $1, $2, $4, $5, $7, $9}' ${FILENAME} >> ${OUTFILE}
  else
    echo Cannot find ${FILENAME}
    exit 1
fi
\end{lstlisting}

Notice that this time we didn't require a file to be given to the script.
We defined it within the script, as we did for the output file.

\begin{information}
The directory \& file checking stages were of the form if [...].
This is a curious command that checks for the presence of something. 
The options -d \& -a specify a directory or file respectively.
\end{information}

\begin{questions}
Will the above script generate a tab, comma or space delimited text file? \\
\begin{answer}
It will be space delimited. 
We could have specified tab delimited by inserting ``\textbackslash t'' between each field.
\end{answer}
\end{questions}

\begin{steps}
Open the \texttt{gedit} text editor \& save the blank file in your directory as \textit{extract\_CDS.sh}.
Now write this above script into the editor, but \textit{taking care to use the directory where you have the .gff file stored in the appropriate place.}
Once you have written the script, save it \& close it.
Now make it executable and run it.\\
\end{steps}


\section{Some Scripting Challenges}
Now that we have written our first couple of scripts, the challenge for the rest of the session will to be create your own script from scratch.
You can use the gedit text editor for this \& it will automatically colour code as you go, which can be very helpful for seeing where you are as you write code.

\begin{advanced}
The restriction site for the GBS-Seq dataset we have is TGCAG.
In the \texttt{~/trainingData} folder, we have both pairs of reads.
Your challenge is to write a script to find:
\begin{enumerate}
\item How many reads there are in each .fq file
\item How many reads begin with the restriction site
\item Output this into a new file, with appropriate column headings. \\
\end{enumerate}

The .fq files we have were processed by a program that has done strange things to the identifiers.
Each piece of information is separated by underscores, which should be colons, as specified by the defined fastq format \url{https://en.wikipedia.org/wiki/FASTQ\_format#Illumina\_sequence\_identifiers}.
The digit indicating which member of a pair the sequence belongs to should also be of a different format, as seen in the given definition.\\

Can you write a script to change these identifiers back to the correct format? 
You can omit the machine identifier, or just make a name up if you feel like it.
\end{advanced}

\section{Moving towards High Performance Computing}
\subsection{Environment Variables}
\begin{information}
Your computer operating system has many variables (i.e. Environment Variables) which tell it how to perform certain tasks.
One of these variables is called \texttt{PATH}, and it is a set of directories that tell your computer where to look for things (like programs).
For example, the command \texttt{ls} actually calls a program called \texttt{ls} which is in one of the directories in your system \texttt{PATH}.\\
\end{information}

\begin{steps}
To add the directory \texttt{~/bin} to your \texttt{PATH} variable in bash, we would first create this directory
\begin{lstlisting}
cd
mkdir bin
\end{lstlisting}
\end{steps}

\begin{steps}
Directories called \texttt{bin} are often when executables (or binaries) are stored.
This is just another way to refer to programs.
If we wanted to place any executable programs in this directory, we could now add this directory to our \texttt{PATH}, and we could call the program simply by name.
Otherwise, we would have to tell our computer exactly where to look for it.

\begin{lstlisting}
export PATH=/home/trainee/bin:$PATH
\end{lstlisting}
\end{steps}

\subsection{High Performance Computing}
\begin{information}
In current genomics era, where we regularly work with large datasets, the amount of resources available on desktop computers are often insufficient to enable your script finish in a reasonable time. 
For these datasets, it maybe useful to work on a high performance computing system, which will enable your data or command to be run simultaneously on $>8$ threads, i.e. in parallel.
It is possible to gain access to large computing resources through eResearch SA \url{https://www.ersa.edu.au}, enabling analysis of large datasets efficiently.\\ 

Their large machine called \texttt{tizard} does not run using the \texttt{bash} shell, instead using a different command-line shell called \texttt{csh}, or the C shell (cue a series of bad puns).
This shell is very similar to \texttt{bash}, but has some slightly different syntax elements, much like different programming languages can have numerous similarities.\\
\end{information}

To set your path in \texttt{csh}, the syntax would be slightly different than for \texttt{bash}.
In this case we would enter \texttt{setenv PATH \$PATH:/home/trainee/bin}.

\begin{information}
Having many users on one machine at one time also means that there needs to be a system which determines who runs what and when. 
Tizard uses a "queuing" system, whereby users submit jobs to a queue, and then executed when the appropriate resources on the machine become available. 
The command is executed using a Portable Batch System (PBS) script (shown below). 
This script is written in \texttt{csh} (not \texttt{bash}) and contains extra parameters such as:

\begin{itemize}
\item Wall Time (The maximum time it is allowed to takes to completion)
\item Amount of memory to allocate to the job
\item Number of "nodes". 
\end{itemize}

\begin{lstlisting}
#!/bin/csh
#PBS -N Squamates_RLC_Uni20_Log1_1
#PBS -j oe
#PBS -m ae
#PBS -M your.name@adelaide.edu.au
#PBS -q tizard
#PBS -l nodes=1:ppn=1
#PBS -l mem=8000mb
#PBS -l vmem=8000mb
#PBS -l walltime=100:00:00

echo Working dir is $PBS_O_WORKDIR
cd $PBS_O_WORKDIR
echo Running on host `hostname`
echo Time is `date`

module load java/java-1.7.09
module load beast/1.8.0
module load beaglebeast/readonlylib-1.0

java -Xms128m -Xmx4096m  -jar /opt/shared/beast/1.8.0/lib/beast.jar -overwrite -beagle -seed 111111111 Squamates_RLC_Uni20_Log1.xml > Squamates_RLC_Uni20_Log1-$PBS_JOBID.output
\end{lstlisting}

We don't need to write this script.
It is included here as a simple example of a real world script as used by researchers at ACAD.
This script may look a little intimidating at first, but slowly work your way through each line \& try to understand what each line is specifying.
This particular script will allow users to submit a BEAST job (Bayesian analysis of molecular sequences using MCMC) to tizard, and will email when its done.
Also note that many of the commands will now look very familiar to you and in essence it is no more complex than the scripts we have written in the earlier sections.
\end{information}
