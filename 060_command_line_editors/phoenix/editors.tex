\setModuleTitle{Command Line Editors}
\setModuleAuthors{%
  Robert Qiao, Research Services Phoenix Team, University of Adelaide
\mailto{robert.qiao@adelaide.edu.au}\\ 
}

%----------------------------------------------------------------------------------------
% MODULE TITLE PAGE
%----------------------------------------------------------------------------------------
% BEGIN: Module Title Page
%  * The chapter page will always appear on odd numbered page
\chapter{\moduleTitle}
\newpage

In addition to the streaming editor named sed as you have studied earlier, which operates on a file
in a non-interactive mode according to a set of instructions that you specify on the command line or
in a
script. It can be used for "canned" editing tasks that must be applied in the same way to many
files. More commonly, we use interactive editors and there are three powerful interactive text
editors that
are commonly used on Unix computers: vi, pico, and emacs. \\

Although vi, pico, emacs are extremely powerful, to get a firm grasp takes some efforts and most of
times we just need to open, amend and save a file without remembering the keyboard shortcuts.
Luckly, there is such editor called nano and that is what we will use for the class today.
The nano editor has its own set of keyboard shortcuts of course and in this guide I aim to help you
to understand the meaning of all those special keystrokes you can use to make your life easier when
using nano. \\

But before we dive into the nano, let's generally go over some of the streamline editors like vi,
emacs. 

\section{vi}

\begin{information}
Vi (pronounced "vee-eye") is the standard editor for Unix systems. It is universally available on
Unix systems. Vi is a screen editor. It treats your computer screen as a window into the file. You
move the window around to view different parts of the file. You move the cursor to the location on
the screen where you want to make a change; or optionally, you specify some kind of global change.
Vi updates your screen to reflect changes that you make in the file. Actually, it works on a copy of
the file in memory, and only updates the file on disk when you tell it to, such as when you end the
editing session. \\

Main advantages of vi are including: 
\begin{itemize}
  \item Vi is universally available on Unix systems. It has been around so long in a stable form
that it is essentially bug free. Many clones have been written for other kinds of computers.
  \item Vi has many powerful commands that utilize just the alphanumeric keys -- it does not require
special function keys.
  \item Vi is a small program that does not require a lot of system memory or CPU time. It works
very fast, even on large files.
  \item While vi is not programmable, it has a simple way to let other Unix programs, such as the
sort utility, work on selected portions of your file. This adds the functionality of all those
programs to the editor.
  \item Vi is completely terminal device independent. It will work with any kind of terminal.
A system file describes the capabilities and control sequences of each kind of terminal for vi. All
the program needs to know is what type of terminal you have. When you log in, if pangea cannot
figure out what kind of terminal you have, it will prompt you to specify a terminal type. The most
common type is the vt100, which most modern terminals and PC communications software emulate.
\end{itemize}

The chief disadvantage of vi is that it is touchy. That is, every single key you touch on the
keyboard seems to do something, often something mysterious. There is a rich set of single character
commands to learn.
\end{information} 

\section{emacs}

\begin{information}
Emacs is a widely used editor available on many types of systems. The GNU version is installed on
pangea. This editor is very large and fully programmable with a built-in object oriented language.
As a result, it has become more of a shell than an editor. Using "macros", it is possible to do many
non-editing functions from within emacs, including compiling and debugging programs and browsing the
web. \\

Emacs is not described in these notes because it is too complicated and bloated for simple editing.
Entire books are available to teach emacs. \\

If you are already familiar with emacs, you can use it on pangea and most other Unix systems on
campus. Simply type the shell command
\begin{lstlisting}
emacs
\end{lstlisting}
\end{information}

\section{Nano}

Now, let's have a more detailed look of Nano. \\

\begin{warning}
To setup this part of the course, please login to your Phoenix account and download the example file
to your /fastdir directory now
\begin{lstlisting}
cp -r /apps/examples/training_linux/ ~/fastdir/
\end{lstlisting}
please check to make sure you command blow contains valid 4 files. If you see error messages, please
indicate to tutors, 
you should get this step fixed before carry on further
\begin{lstlisting}
ls -al ~/fastdir/training_linux
\end{lstlisting}
\end{warning}

\subsection{How To Get Nano}
The nano editor is available by default in all the most popular Linux distributions and you can run
it with one simple command:
\begin{lstlisting}
nano 
\end{lstlisting}
The above command will simply open a new file. You can type into the window, save the file and exit.

\subsection{How To Open A New File And Give It A Name Using Nano}
Whilst simply running nano is ok you might want to give your document a name before starting. To do
this simply give the filename after the nano command.
\begin{lstlisting}
nano myfile.txt
\end{lstlisting}

You can, of course, supply a complete path to create a file anywhere on your Linux system (as long
as you have the permissions to do so).
\begin{lstlisting}
nano ~/fastdir/training_linux/myfile.txt
\end{lstlisting}

\subsection{How To Open An Existing File Using Nano}

You can use the same command as the one above to open an existing file.

Simply run nano with the path to the file you wish to open.

To be able to edit the file you must have permissions to edit the file otherwise, it will open as a
readonly file (assuming you have read permissions).

\begin{lstlisting}
cd ~/fastdir/training_linux
nano seqData.fastq
\end{lstlisting}

You can, of course, supply a complete path to open a file anywhere on your Linux system (as long as
you have the permissions to do so).
\begin{lstlisting}
nano ~/fastdir/training_linux/seqData.fastq
\end{lstlisting}

\subsection{How To Save A File Using Nano}
You can add text to the nano editor simply by typing the contents directly into the editor. Saving
the file, however, requires the use of a keyboard shortcut.

To save a file in nano press \texttt{ctrl} and \texttt{O} at the same time.

If your file already has a name you just need to press enter to confirm the name otherwise you will
need to enter the filename that you wish to save the file as.

\subsubsection{How To Save A File In DOS Format Using Nano}
To save the file in DOS format press \texttt{ctrl}  and \texttt{O} to bring up the filename box. Now
press \texttt{alt} and \texttt{d} for DOS format.
\subsubsection{How To Save A File In MAC Format Using Nano}
To save the file in MAC format press \texttt{ctrl}  and \texttt{O} to bring up the filename box. Now
press \texttt{alt} and \texttt{m}  for MAC format.

\subsection{How To Backup A File Before Saving It In Nano}
If you want to save the changes to a file that you are editing but you want to keep a backup of the
original press \texttt{ctrl}  and \texttt{O} to bring up the save window and then press \texttt{alt}
and \texttt{B}.

The word [backup] will appear in the filename box.

After you have finished editing your file you will want to leave the nano editor.

To exit nano simply press \texttt{ctrl}  and \texttt{x} at the same time.

If the file hasn't been saved you will be prompted to do so. If you select "Y" then you will be
prompted to enter a file name.

\subsection{Displaying Cursor Position Using Nano}
If you wish to know how far down a document you are within nano you can press the \texttt{ctrl}  and
\texttt{c} keys at the same time.

The output is shown in the following format:
\begin{lstlisting}
line 5/11 (54%), col  10/100 (10%),  char 100/200 (50%)
\end{lstlisting}

\subsection{How To Read A File Using Nano}
If you opened nano without specifying a filename you can open a file by pressing \texttt{ctrl}  and
\texttt{r} at the same time.

You are now able to specify a filename to read into the editor. If you already have text loaded into
the window the file you read in will append itself to the bottom of your current text.

If you want to open the new file in a new buffer press \texttt{alt} and \texttt{f}.

\subsection{How To Search And Replace Using Nano}
To start a search within nano press \texttt{ctrl}  and \texttt{\\}. 

To turn off replace press \texttt{ctrl}  and \texttt{r}. You can turn on replace again by repeating
the keystroke.

To search for text enter the text you wish to search for and press return.

To search backwards through the file press \texttt{ctrl}  and \texttt{r} to bring up the search
window. Press \texttt{alt} and \texttt{b}.

To force case sensitivity bring up the search window again and then press \texttt{alt} and
\texttt{c}. You can turn it off again by repeating the keystroke.

Nano wouldn't be a Linux text editor if it didn't provide a way to search using regular expressions.
To turn regular expressions on bring up the search window again and then press alt and r.

You can now use regular expressions for searching for text.

\subsection{Check Your Spelling Within Nano}
Again nano is a text editor and not a word processor so I'm not sure why spelling is a key feature
of it but you can indeed check your spellings using the \texttt{ctrl}  and \texttt{t} keyboard shortcut.

\subsection{Some Goodies}
\begin{information}
There are a number of switches you can specify when using nano. The best ones are covered below. You can find the rest by reading the nano manual.
\begin{itemize}
	\item nano -B (backs up the file prior to editing it)  
	\item nano -E (converts tabs to spaces when editing)
	\item nano -c (constantly show the cursor position stats)
	\item nano -i (automatically indents new lines to the same position as the previous line)
	\item nano -k (toggle cut so that it cuts from cursor position instead of the whole line)
	\item nano -m (provides mouse support to the editor)
	\item nano -v (opens file as readonly)
\end{itemize}
\end{information}


